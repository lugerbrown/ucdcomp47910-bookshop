\documentclass[]{UCD_CS_FYP_Report}
\usepackage{graphicx}
\usepackage{hyperref}

%%%%%%%%%%%%%%%%%%%%%%
%%% Input project details

\def\studentname{Luis Marron (23202882)} % Edit with your name
\def\projecttitle{{\linespread{4.5}\selectfont COMP47910 Secure Software Engineering}} % Edit with you project title
\def\supervisorname{Dr. Liliana Pasquale} % Edit with your supervisor name


\begin{document}

\maketitle

%%%%%%%%%%%%%%%%%%%%%%
%%% Table of Content

\tableofcontents\pdfbookmark[0]{Table of Contents}{toc}\newpage
\newpage

%%%%%%%%%%%%%%%%%%%%%%
%%% Vulnerabilities Fixes Report


\chapter{A01:2021 Broken Access Control}


\section{CWE-284: Improper Access Control}

\textbf{Description}: The BookShop application implements improper access control mechanisms that allow unauthorized users to access restricted functionality and sensitive data. The application lacks proper authentication and authorization checks across multiple endpoints, enabling attackers to bypass security controls and gain unauthorized access to administrative functions and user-specific resources.

\textbf{CWE Explanation}: CWE-284 occurs when the application fails to properly restrict access to functionality or resources, allowing unauthorized users to perform actions or access data that should be protected by proper authentication and authorization mechanisms.

\textbf{Severity}: High

\subsection*{Vulnerability Locations and Type}
The application suffered from multiple \textit{Broken / Improper Access Control} weaknesses manifesting as:
\begin{itemize}
	\item \textbf{Vertical privilege escalation}: Security rule protected path pattern \texttt{/admin/**}, while the actual admin endpoint controller was mapped to \texttt{/admins}. This mismatch let any authenticated non-admin invoke administrative endpoints (e.g. \texttt{GET /admins}).
	\item \textbf{Horizontal privilege escalation / Insecure Direct Object Reference (IDOR)}: Cart- and cart-item–related REST endpoints (\texttt{/carts/**}, \texttt{/cart-items/**}) accepted arbitrary \texttt{customerId}, \texttt{cartId}, or \texttt{itemId} without verifying ownership against the authenticated principal, enabling access/modification of other users' carts.
	\item \textbf{Over-broad data exposure}: User and customer enumeration endpoints (\texttt{/users/**}, \texttt{/customers/**}) allowed any authenticated user to list or fetch other users/customers and (in the case of \texttt{/users}) returned password hashes, breaching least privilege and confidentiality.
	\item \textbf{Unrestricted modification endpoints}: Book management API (\texttt{/api/books} POST/PUT/DELETE) was accessible to any authenticated role, permitting non-admin content manipulation.
	\item \textbf{Lack of server-side authorization on web cart actions}: Web MVC endpoints for cart item removal did not assert ownership, allowing crafted requests to remove other users' items.
\end{itemize}

\subsection*{Mitigation Strategy and Rationale}
Defence-in-depth was applied combining \textbf{path-based authorization}, \textbf{method-level role enforcement}, and \textbf{resource ownership (row-level) checks}. This layered approach ensures:
\begin{itemize}
	\item Misconfigurations in one layer (e.g. path patterns) do not automatically grant access because method-level annotations add a second gate.
	\item Even with correct role checks, horizontal attacks (guessing IDs) are blocked by explicit ownership validation tying domain object identifiers to the authenticated principal (prevents IDOR).
	\item Principle of Least Privilege is restored by scoping sensitive endpoints (admin, user lists, data mutations) strictly to the required roles, reducing attack surface.
	\item Sensitive credential fields are no longer exposed after authorization succeeds, limiting post-auth data leakage channels.
\end{itemize}

\subsection*{Implemented Security Controls}
The following concrete changes were introduced in code (branch \texttt{broken-access-control-fixes}):
\begin{enumerate}
	\item \textbf{Corrected admin path mapping}: Updated Spring Security configuration to match the actual controller path \texttt{/admins/**} instead of the incorrect \texttt{/admin/**}.
	\item \textbf{Granular authorization rules}: Added HTTP method specific matchers restricting book mutation endpoints (POST/PUT/DELETE on \texttt{/api/books/**}) and all \texttt{/books/**} (management pages) to role \texttt{ADMIN}. User endpoints \texttt{/users/**} now require \texttt{ADMIN}; customer endpoints enforce authentication plus ownership (see point 3) and reserve modifications for \texttt{ADMIN}.
	\item \textbf{Ownership / row-level checks}: Injected \texttt{@AuthenticationPrincipal} into cart and cart item controllers; added helper methods validating that the referenced cart/item's owning customer's username equals the authenticated principal. Requests failing validation now return HTTP 403 (or 404 when appropriate).
	\item \textbf{Method-level security}: Enabled \texttt{@EnableMethodSecurity} and introduced \texttt{@PreAuthorize} annotations on controller methods (e.g. book mutations, user controller) to provide a second authorization layer beyond URL pattern matching.
	\item \textbf{Data minimisation via DTOs}: Replaced direct entity exposure for users and customers with DTOs omitting the password field and limiting attributes to non-sensitive identification data.
	\item \textbf{Cart web endpoint hardening}: Added ownership validation before allowing removal of a cart item via web MVC endpoint to block cross-user manipulation.
	\item \textbf{Input encapsulation}: Refactored mutable request payload classes (e.g. add-item request) to use private fields with accessors, preventing accidental uncontrolled field exposure (minor hardening).
\end{enumerate}

\subsection*{Solution Effectiveness}
The mitigations directly address the CWE-284 root causes:
\begin{itemize}
	\item Path correction + role restrictions close vertical privilege escalation vectors.
	\item Ownership validation eliminates horizontal ID enumeration attacks by binding operations to the authenticated identity.
	\item Method-level annotations safeguard against future path mapping drift or overly broad ant matchers.
	\item DTO-based redaction removes unnecessary sensitive data from responses, shrinking impact radius of any residual access gaps.
	\item Principle of Least Privilege is enforced consistently across both REST and MVC layers, aligning actual access with business intent.
\end{itemize}



\section{CWE-639: Insecure Direct Object References }

	extbf{Description}: Multiple endpoints previously trusted user-supplied identifiers (e.g. \texttt{customerId}, \texttt{cartId}, \texttt{itemId}) directly to select resources without verifying that the authenticated principal owned or was entitled to those resources. Attackers could substitute another valid numeric ID (IDOR) to read or modify other users' shopping cart contents or personal data.

	extbf{CWE Explanation}: CWE-639 occurs when an application authorizes a request based solely on a user-controlled key (such as a record ID) instead of confirming the requester is permitted to access the referenced object, enabling horizontal privilege escalation (Insecure Direct Object Reference).

	extbf{Severity}: High

\subsection*{Vulnerability Locations and Type}
\begin{itemize}
	\item \textbf{Cart REST endpoints}: \texttt{/carts/by-customer/{customerId}}, \texttt{/carts/{cartId}/items}, \texttt{/carts/{cartId}/add-item}, \texttt{/carts/{cartId}/remove-item/{itemId}}, and \texttt{/carts/{cartId}/total-price} accepted arbitrary path IDs; prior to fixes there was no binding between these IDs and the authenticated user (classic IDOR / horizontal escalation).
	\item \textbf{Cart item endpoint}: \texttt{/cart-items/{id}} exposed individual cart items without verifying ownership (permitting enumeration of other users' items by ID).\newline
	\item \textbf{Customer lookup endpoints}: \texttt{/customers/{id}} and \texttt{/customers/by-username/{username}} allowed retrieval of other customers' personal information without ownership check (in early state) relying only on being authenticated.
\end{itemize}

\subsection*{Mitigation Strategy and Rationale}
The strategy focused on eliminating trust in client-supplied identifiers by: (1) performing \textbf{server-side ownership validation} after object retrieval, (2) layering \textbf{role checks} so only admins can enumerate accounts, and (3) reducing exposed data via DTOs. Ownership checks ensure an attacker who guesses an ID still receives \texttt{403 Forbidden} (or \texttt{404 Not Found}) unless authorized. This halts horizontal privilege escalation while preserving legitimate functionality for rightful owners.

\subsection*{Implemented Security Controls}
\begin{enumerate}
	\item \textbf{Ownership helpers}: Introduced methods \texttt{enforceCustomerOwnership(...)} and \texttt{resolvedOwnedCart(...)} in the cart controller to load the resource and verify \texttt{resource.owner.username == principal.username} before returning it.
	\item \textbf{Cart item deletion hardening}: Added explicit retrieval and cart-to-item linkage verification before deleting an item, ensuring the item belongs to the authenticated user's cart, preventing cross-cart deletions.
	\item \textbf{Principal injection}: Added \texttt{@AuthenticationPrincipal} parameters to affected controller methods to reliably access the authenticated identity rather than relying on user-provided IDs.
	\item \textbf{Customer data protection}: Added central ownership/admin check (\texttt{enforceOwnershipOrAdmin}) to customer endpoints and restricted full listing to \texttt{ADMIN} via \texttt{@PreAuthorize}.
	\item \textbf{Least privilege for enumeration}: Restricted \texttt{/users/**} and customer listing endpoints to \texttt{ADMIN} in \texttt{SecurityConfig}, removing the ability for regular users to guess IDs and enumerate.
	\item \textbf{DTO redaction}: Replaced direct entity exposure with DTOs omitting password hashes, limiting the value of any accidental disclosure.
	\item \textbf{Path-based + method security}: Retained path restrictions (role-based) and enabled method-level security, adding a secondary barrier should future path patterns broaden inadvertently.
\end{enumerate}

\subsection*{Solution Effectiveness}
\begin{itemize}
	\item Ownership enforcement turns previously unauthenticated resource selection into a two-factor authorization (valid ID + rightful owner) neutralising ID guessing.
	\item Removal operations now validate item-to-cart ownership, closing a residual destructive IDOR vector.
	\item Admin-only enumeration eliminates bulk discovery of valid identifiers (reducing reconnaissance surface).
	\item DTO redaction ensures that even if an ownership check regresses, sensitive credential data remains undisclosed.
	\item Layered (config + method) checks reduce single-point-of-failure risk if URL matcher misconfiguration reappears.
\end{itemize}




\chapter{A02:2021 Cryptographic Failures}

\section{CWE-326: Inadequate Encryption Strength}

\textbf{Description}: The BookShop application stores user passwords in plain text without any cryptographic protection. The database contains user credentials including admin passwords stored as clear text, making them immediately readable if the database is compromised. The application also lacks any password hashing or encryption mechanisms in the authentication process.

\textbf{CWE Explanation}: CWE-326 occurs when the application uses cryptographic algorithms or key sizes that are insufficient to protect sensitive data, making it vulnerable to brute force attacks and unauthorized access.

\textbf{Severity}: Critical



\section{CWE-522: Insufficiently Protected Credentials}

\textbf{Description}: The BookShop application uses Spring Security framework but fails to implement proper credential protection mechanisms. While the framework provides password hashing and encoding capabilities, the application stores passwords in plain text and performs direct string comparisons without any hashing, salting, or other protection mechanisms.

\textbf{CWE Explanation}: CWE-522 occurs when the application uses a protection mechanism that is insufficient to protect credentials, such as using weak hashing algorithms, not using salt, or failing to implement available security features properly.

\textbf{Severity}: Critical



\chapter{A03:2021 Injection}

\section{CWE-79: Cross-Site Scripting (XSS)}

\textbf{Description}: The BookShop application was analyzed for Cross-Site Scripting (XSS) vulnerabilities due to improper handling of user-controlled data in error messages and direct rendering of user input without proper sanitization or encoding.

\textbf{CWE Explanation}: CWE-79 occurs when the application fails to properly validate, sanitize, or encode user-controlled input before including it in output that is sent to other users' browsers, allowing attackers to execute malicious scripts in the context of other users' sessions.

\textbf{Severity}: Low



\section{CWE-190: Integer Overflow or Wraparound}

\textbf{Description}: The BookShop application is vulnerable to integer overflow and underflow attacks due to improper handling of numeric operations without proper validation or overflow checks. This vulnerability allows attackers to manipulate business logic by providing malicious numeric values that cause integer wraparound.

\textbf{CWE Explanation}: CWE-190 occurs when the application performs arithmetic operations on integers without checking for overflow or underflow conditions, allowing attackers to manipulate numeric values to cause unexpected behavior, data corruption, or bypass business logic controls.

\textbf{Severity}: Medium



\chapter{A04:2021 Insecure Design}

\section{CWE-602: Client-Side Enforcement of Server-Side Security}

\textbf{Description}: The BookShop application implements critical security controls on the client-side instead of the server-side, allowing attackers to bypass authentication, authorization, and input validation by making direct API calls to backend endpoints. The application relies on client-side JavaScript to enforce security policies that should be implemented on the server.

\textbf{CWE Explanation}: CWE-602 occurs when the application implements security controls (authentication, authorization, input validation) on the client-side rather than the server-side, making them easily bypassable by attackers who can make direct HTTP requests to backend endpoints.

\textbf{Severity}: High



\section{CWE-799: Improper Control of Interaction Frequency}

\textbf{Description}: The BookShop application lacks any rate limiting or frequency control mechanisms, allowing unlimited interaction attempts with all endpoints. The application does not implement authentication attempt limits, request throttling, or any protection against brute force attacks, making it vulnerable to automated attacks and resource exhaustion.

\textbf{CWE Explanation}: CWE-799 occurs when the application fails to properly control the frequency of interactions, allowing attackers to make unlimited requests that can lead to brute force attacks, resource exhaustion, and denial of service conditions.

\textbf{Severity}: High



\section{CWE-840: Business Logic Errors}

\textbf{Description}: The BookShop application contains multiple business logic errors that violate fundamental application rules and constraints. These include race conditions in the checkout process, lack of duplicate username validation, client-side price calculation vulnerabilities, and missing order validation rules that can lead to inventory overselling, account confusion, and financial manipulation.

\textbf{CWE Explanation}: CWE-840 occurs when the application fails to properly implement business rules and constraints, allowing attackers to exploit logical flaws in the application's workflow, data validation, and state management to achieve unauthorized outcomes.

\textbf{Severity}: High



\section{CWE-1173: Improper Use of Validation Framework}

\textbf{Description}: The BookShop application completely lacks proper validation framework usage, with no Bean Validation annotations, no @Valid annotations in controllers, and no validation framework dependencies. The application relies solely on minimal client-side validation, allowing malicious or invalid input to bypass security controls and potentially cause data integrity issues, application instability, and security vulnerabilities.

\textbf{CWE Explanation}: CWE-1173 occurs when the application fails to properly use validation frameworks or implements validation incorrectly, allowing invalid or malicious input to bypass security controls, potentially leading to data integrity issues, application instability, and security vulnerabilities.

\textbf{Severity}: High




\chapter{A05:2021 Security Misconfiguration}

\section{CWE-614: Sensitive Cookie in HTTPS Session Without 'Secure' Attribute}

\textbf{Description}: The BookShop application's session cookies (JSESSIONID) are transmitted without the 'Secure' attribute, allowing them to be sent over unencrypted HTTP connections. This vulnerability exposes session tokens to potential interception by attackers through network sniffing, man-in-the-middle attacks, or other network-based attacks.

\textbf{CWE Explanation}: CWE-614 occurs when sensitive cookies are transmitted over insecure channels without proper protection mechanisms, allowing attackers to capture and reuse session tokens to impersonate authenticated users.

\textbf{Severity}: High



\section{CWE-1275: Sensitive Cookie with Improper SameSite Attribute}

\textbf{Description}: The BookShop application's session cookies lack proper SameSite attribute configuration, allowing them to be sent in cross-site requests and enabling various client-side attacks including CSRF.

\textbf{CWE Explanation}: CWE-1275 occurs when cookies are configured without appropriate SameSite restrictions, allowing them to be sent in cross-site requests and enabling various client-side attacks including CSRF.

\textbf{Severity}: High



\section{CWE-693: Protection Mechanism Failure}

\textbf{Description}: The BookShop application includes Spring Security framework but completely disables all protection mechanisms, rendering the security framework ineffective. The application disables CSRF protection, permits all requests without authentication, and fails to implement any of the available security controls, making the protection mechanism completely non-functional.

\textbf{CWE Explanation}: CWE-693 occurs when the application has a protection mechanism in place but fails to use it properly, rendering the security controls ineffective and leaving the application vulnerable to attacks that the protection mechanism was designed to prevent.

\textbf{Severity}: High



\section{CWE-256: Unprotected Storage of Credentials}

\textbf{Description}: The BookShop application stores credentials in unprotected form across multiple locations including configuration files, database initialization scripts, and Docker environment variables. All credentials are stored in plain text without any encryption, hashing, or other protection mechanisms.

\textbf{CWE Explanation}: CWE-256 occurs when the application stores sensitive credentials (passwords, keys, tokens) without proper protection mechanisms, making them vulnerable to unauthorized access and compromise.

\textbf{Severity}: Critical



\chapter{A06:2021 Vulnerable and Outdated Components}

\section{CWE-269: Improper Privilege Management}

\textbf{Description}: The BookShop application exhibits multiple critical privilege management failures including missing authentication for critical admin functions, inconsistent privilege enforcement across endpoints, complete bypass of Spring Security framework, and lack of ownership verification for user resources. These vulnerabilities enable unauthorized access, privilege escalation, and cross-user data manipulation.

\textbf{CWE Explanation}: CWE-269 occurs when the application fails to properly manage privileges, permissions, and access controls, allowing unauthorized users to access restricted functionality or resources that should be protected by proper authentication and authorization mechanisms.

\textbf{Severity}: Critical



\section{CWE-400: Uncontrolled Resource Consumption}

\textbf{Description}: The BookShop application lacks proper resource management controls including database connection pool limits, session timeout limits, request timeout limits, and memory limits. This vulnerability allows attackers to exhaust system resources through unlimited requests, session creation, and large payload attacks, potentially leading to denial of service conditions.

\textbf{CWE Explanation}: CWE-400 occurs when the application fails to properly control resource consumption, allowing attackers to exhaust system resources such as memory, CPU, database connections, or network bandwidth through unlimited or uncontrolled operations, leading to denial of service conditions.

\textbf{Severity}: High



\chapter{A07:2021 Identification and Authentication Failures}

\section{CWE-287: Improper Authentication}

\textbf{Description}: The BookShop application implements fundamentally flawed authentication mechanisms including plain text password storage, direct password comparison without hashing, weak password policies, and complete absence of authentication security controls. The application stores user credentials in plain text in the database and performs direct string comparison during login, making it vulnerable to complete account compromise and unauthorized access.

\textbf{CWE Explanation}: CWE-287 occurs when the application fails to properly verify the identity of users, implement secure authentication mechanisms, or protect authentication credentials, allowing attackers to bypass authentication controls, compromise user accounts, and gain unauthorized access to sensitive data and functionality.

\textbf{Severity}: Critical



\section{CWE-384: Session Fixation}

	extbf{Description}: Initially the session fixation risk was assessed because the configuration did not explicitly state a session fixation protection strategy, and CSRF protection was disabled (increasing the impact of any session hijacking). However, Spring Security's default behaviour already migrates (rotates) the session identifier upon successful authentication. We have now made this protection explicit in the security configuration, confirming that an attacker who pre-seeds a victim with a known anonymous session ID cannot continue to use that same ID after the victim authenticates.

	extbf{CWE Explanation}: CWE-384 (Session Fixation) concerns failure to invalidate or regenerate a session identifier when a user's authentication state changes, enabling an attacker who knows the pre-auth session token to hijack the post-auth session.

	extbf{Severity}: Low (Residual) – Effective mitigations in place; only hardening items remain.

\subsection*{Vulnerability Locations and Type}
Prior to the explicit mitigation the potential exposure was theoretical rather than an observed exploit, characterised by:
\begin{itemize}
	\item No explicit session fixation directive in \texttt{SecurityConfig} (relying implicitly on framework defaults).\newline
	\item Disabled CSRF protection (would amplify damage of any successful fixation / hijack).\newline
	\item Absence of cookie security attribute configuration (Secure / SameSite / HttpOnly not yet declared in \texttt{application.properties}).
\end{itemize}
Crucially, there was \emph{no} custom code overriding Spring Security's default fixation protection; thus the core exploit path (reusing the same session ID after login) was already blocked.

\subsection*{Mitigation Strategy and Rationale}
Strategy focused on: (1) making implicit framework protections \emph{explicit} for auditability, (2) reducing residual attack surface through planned cookie hardening and CSRF re-enablement, and (3) documenting session handling to prevent future regressions (e.g. switching to a stateless policy without compensating controls). Explicit configuration eliminates uncertainty for reviewers and compliance checks.

\subsection*{Implemented Security Controls}
\begin{enumerate}
	\item \textbf{Explicit session ID migration}: Added \texttt{sessionFixation(migrateSession())} under \texttt{sessionManagement} in \texttt{SecurityConfig}, guaranteeing a new session identifier post-authentication.
	\item \textbf{Logout invalidation}: Existing logout configuration invalidates the session server-side, preventing reuse of an authenticated context.
	\item \textbf{Least privilege hardening elsewhere}: Reduced horizontal/vertical escalation (CWE-284 / CWE-639 controls) lowers the value of any hypothetical hijacked session.
	\item \textbf{Password hashing (BCrypt)}: Limits credential replay even if a session were short-livedly exposed.
\end{enumerate}

\subsection*{Solution Effectiveness}
\begin{itemize}
	\item Attacker-controlled pre-auth session IDs are invalidated because the server issues a fresh session ID upon authentication (session fixation vector neutralised).
	\item Logout and privilege boundaries ensure compromised sessions cannot silently escalate or persist indefinitely.
	\item Strengthened access controls reduce the actionable scope of any transient session misuse.
	\item Making the protection explicit aids code reviews and prevents accidental regression (e.g. future refactor removing defaults unnoticed).
\end{itemize}

Given the default and now explicit migration behaviour, the application is not meaningfully vulnerable to session fixation at present; residual items are advisory enhancements.



\end{document}
