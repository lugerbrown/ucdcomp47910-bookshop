\documentclass[]{UCD_CS_FYP_Report}
\usepackage{graphicx}
\usepackage{hyperref}

%%%%%%%%%%%%%%%%%%%%%%
%%% Input project details

\def\studentname{Luis Marron (23202882)} % Edit with your name
\def\projecttitle{{\linespread{4.5}\selectfont COMP47910 Secure Software Engineering}} % Edit with you project title
\def\supervisorname{Dr. Liliana Pasquale} % Edit with your supervisor name


\begin{document}

\maketitle

%%%%%%%%%%%%%%%%%%%%%%
%%% Table of Content

\tableofcontents\pdfbookmark[0]{Table of Contents}{toc}\newpage
\newpage

%%%%%%%%%%%%%%%%%%%%%%
%%% Vulnerabilities Fixes Report


\chapter{A01:2021 Broken Access Control}


\section{CWE-306: Missing Authentication for Critical Function}

\textbf{Description}: The BookShop application has a Cross-Origin Resource Sharing (CORS) misconfiguration that permits cross-domain read requests from arbitrary third-party domains to unauthenticated APIs. While web browser implementations do not permit arbitrary third parties to read responses from authenticated APIs, this misconfiguration could still be exploited by attackers to access sensitive data through unauthenticated endpoints.

\textbf{CWE Explanation}: CWE-264 occurs when the application fails to properly restrict cross-origin requests, allowing unauthorized domains to access resources and potentially sensitive data.

\textbf{Severity}: Medium

\section{CWE-639: Authorization Bypass Through User-Controlled Key}

\textbf{Description}: The BookShop application has a Cross-Origin Resource Sharing (CORS) misconfiguration that permits cross-domain read requests from arbitrary third-party domains to unauthenticated APIs. While web browser implementations do not permit arbitrary third parties to read responses from authenticated APIs, this misconfiguration could still be exploited by attackers to access sensitive data through unauthenticated endpoints.

\textbf{CWE Explanation}: CWE-264 occurs when the application fails to properly restrict cross-origin requests, allowing unauthorized domains to access resources and potentially sensitive data.

\textbf{Severity}: Medium

\section{CWE-264: Permissions, Privileges, and Access Controls}

\textbf{Description}: The BookShop application has a Cross-Origin Resource Sharing (CORS) misconfiguration that permits cross-domain read requests from arbitrary third-party domains to unauthenticated APIs. While web browser implementations do not permit arbitrary third parties to read responses from authenticated APIs, this misconfiguration could still be exploited by attackers to access sensitive data through unauthenticated endpoints.

\textbf{CWE Explanation}: CWE-264 occurs when the application fails to properly restrict cross-origin requests, allowing unauthorized domains to access resources and potentially sensitive data.

\textbf{Severity}: Medium



\chapter{A02:2021 Cryptographic Failures}

\section{CWE-326: Inadequate Encryption Strength}

\textbf{Description}: The BookShop application stores user passwords in plain text without any cryptographic protection. The database contains user credentials including admin passwords stored as clear text, making them immediately readable if the database is compromised. The application also lacks any password hashing or encryption mechanisms in the authentication process.

\textbf{CWE Explanation}: CWE-326 occurs when the application uses cryptographic algorithms or key sizes that are insufficient to protect sensitive data, making it vulnerable to brute force attacks and unauthorized access.

\textbf{Severity}: Critical



\section{CWE-522: Insufficiently Protected Credentials}

\textbf{Description}: The BookShop application uses Spring Security framework but fails to implement proper credential protection mechanisms. While the framework provides password hashing and encoding capabilities, the application stores passwords in plain text and performs direct string comparisons without any hashing, salting, or other protection mechanisms.

\textbf{CWE Explanation}: CWE-522 occurs when the application uses a protection mechanism that is insufficient to protect credentials, such as using weak hashing algorithms, not using salt, or failing to implement available security features properly.

\textbf{Severity}: Critical



\chapter{A03:2021 Injection}

\section{CWE-79: Cross-Site Scripting (XSS)}

\textbf{Description}: The BookShop application was analyzed for Cross-Site Scripting (XSS) vulnerabilities due to improper handling of user-controlled data in error messages and direct rendering of user input without proper sanitization or encoding.

\textbf{CWE Explanation}: CWE-79 occurs when the application fails to properly validate, sanitize, or encode user-controlled input before including it in output that is sent to other users' browsers, allowing attackers to execute malicious scripts in the context of other users' sessions.

\textbf{Severity}: Low



\section{CWE-190: Integer Overflow or Wraparound}

\textbf{Description}: The BookShop application is vulnerable to integer overflow and underflow attacks due to improper handling of numeric operations without proper validation or overflow checks. This vulnerability allows attackers to manipulate business logic by providing malicious numeric values that cause integer wraparound.

\textbf{CWE Explanation}: CWE-190 occurs when the application performs arithmetic operations on integers without checking for overflow or underflow conditions, allowing attackers to manipulate numeric values to cause unexpected behavior, data corruption, or bypass business logic controls.

\textbf{Severity}: Medium



\chapter{A04:2021 Insecure Design}

\section{CWE-602: Client-Side Enforcement of Server-Side Security}

\textbf{Description}: The BookShop application implements critical security controls on the client-side instead of the server-side, allowing attackers to bypass authentication, authorization, and input validation by making direct API calls to backend endpoints. The application relies on client-side JavaScript to enforce security policies that should be implemented on the server.

\textbf{CWE Explanation}: CWE-602 occurs when the application implements security controls (authentication, authorization, input validation) on the client-side rather than the server-side, making them easily bypassable by attackers who can make direct HTTP requests to backend endpoints.

\textbf{Severity}: High



\section{CWE-799: Improper Control of Interaction Frequency}

\textbf{Description}: The BookShop application lacks any rate limiting or frequency control mechanisms, allowing unlimited interaction attempts with all endpoints. The application does not implement authentication attempt limits, request throttling, or any protection against brute force attacks, making it vulnerable to automated attacks and resource exhaustion.

\textbf{CWE Explanation}: CWE-799 occurs when the application fails to properly control the frequency of interactions, allowing attackers to make unlimited requests that can lead to brute force attacks, resource exhaustion, and denial of service conditions.

\textbf{Severity}: High



\section{CWE-840: Business Logic Errors}

\textbf{Description}: The BookShop application contains multiple business logic errors that violate fundamental application rules and constraints. These include race conditions in the checkout process, lack of duplicate username validation, client-side price calculation vulnerabilities, and missing order validation rules that can lead to inventory overselling, account confusion, and financial manipulation.

\textbf{CWE Explanation}: CWE-840 occurs when the application fails to properly implement business rules and constraints, allowing attackers to exploit logical flaws in the application's workflow, data validation, and state management to achieve unauthorized outcomes.

\textbf{Severity}: High



\section{CWE-1173: Improper Use of Validation Framework}

\textbf{Description}: The BookShop application completely lacks proper validation framework usage, with no Bean Validation annotations, no @Valid annotations in controllers, and no validation framework dependencies. The application relies solely on minimal client-side validation, allowing malicious or invalid input to bypass security controls and potentially cause data integrity issues, application instability, and security vulnerabilities.

\textbf{CWE Explanation}: CWE-1173 occurs when the application fails to properly use validation frameworks or implements validation incorrectly, allowing invalid or malicious input to bypass security controls, potentially leading to data integrity issues, application instability, and security vulnerabilities.

\textbf{Severity}: High




\chapter{A05:2021 Security Misconfiguration}

\section{CWE-614: Sensitive Cookie in HTTPS Session Without 'Secure' Attribute}

\textbf{Description}: The BookShop application's session cookies (JSESSIONID) are transmitted without the 'Secure' attribute, allowing them to be sent over unencrypted HTTP connections. This vulnerability exposes session tokens to potential interception by attackers through network sniffing, man-in-the-middle attacks, or other network-based attacks.

\textbf{CWE Explanation}: CWE-614 occurs when sensitive cookies are transmitted over insecure channels without proper protection mechanisms, allowing attackers to capture and reuse session tokens to impersonate authenticated users.

\textbf{Severity}: High



\section{CWE-1275: Sensitive Cookie with Improper SameSite Attribute}

\textbf{Description}: The BookShop application's session cookies lack proper SameSite attribute configuration, allowing them to be sent in cross-site requests and enabling various client-side attacks including CSRF.

\textbf{CWE Explanation}: CWE-1275 occurs when cookies are configured without appropriate SameSite restrictions, allowing them to be sent in cross-site requests and enabling various client-side attacks including CSRF.

\textbf{Severity}: High



\section{CWE-693: Protection Mechanism Failure}

\textbf{Description}: The BookShop application includes Spring Security framework but completely disables all protection mechanisms, rendering the security framework ineffective. The application disables CSRF protection, permits all requests without authentication, and fails to implement any of the available security controls, making the protection mechanism completely non-functional.

\textbf{CWE Explanation}: CWE-693 occurs when the application has a protection mechanism in place but fails to use it properly, rendering the security controls ineffective and leaving the application vulnerable to attacks that the protection mechanism was designed to prevent.

\textbf{Severity}: High



\section{CWE-256: Unprotected Storage of Credentials}

\textbf{Description}: The BookShop application stores credentials in unprotected form across multiple locations including configuration files, database initialization scripts, and Docker environment variables. All credentials are stored in plain text without any encryption, hashing, or other protection mechanisms.

\textbf{CWE Explanation}: CWE-256 occurs when the application stores sensitive credentials (passwords, keys, tokens) without proper protection mechanisms, making them vulnerable to unauthorized access and compromise.

\textbf{Severity}: Critical



\chapter{A06:2021 Vulnerable and Outdated Components}

\section{CWE-269: Improper Privilege Management}

\textbf{Description}: The BookShop application exhibits multiple critical privilege management failures including missing authentication for critical admin functions, inconsistent privilege enforcement across endpoints, complete bypass of Spring Security framework, and lack of ownership verification for user resources. These vulnerabilities enable unauthorized access, privilege escalation, and cross-user data manipulation.

\textbf{CWE Explanation}: CWE-269 occurs when the application fails to properly manage privileges, permissions, and access controls, allowing unauthorized users to access restricted functionality or resources that should be protected by proper authentication and authorization mechanisms.

\textbf{Severity}: Critical



\section{CWE-400: Uncontrolled Resource Consumption}

\textbf{Description}: The BookShop application lacks proper resource management controls including database connection pool limits, session timeout limits, request timeout limits, and memory limits. This vulnerability allows attackers to exhaust system resources through unlimited requests, session creation, and large payload attacks, potentially leading to denial of service conditions.

\textbf{CWE Explanation}: CWE-400 occurs when the application fails to properly control resource consumption, allowing attackers to exhaust system resources such as memory, CPU, database connections, or network bandwidth through unlimited or uncontrolled operations, leading to denial of service conditions.

\textbf{Severity}: High



\chapter{A07:2021 Identification and Authentication Failures}

\section{CWE-287: Improper Authentication}

\textbf{Description}: The BookShop application implements fundamentally flawed authentication mechanisms including plain text password storage, direct password comparison without hashing, weak password policies, and complete absence of authentication security controls. The application stores user credentials in plain text in the database and performs direct string comparison during login, making it vulnerable to complete account compromise and unauthorized access.

\textbf{CWE Explanation}: CWE-287 occurs when the application fails to properly verify the identity of users, implement secure authentication mechanisms, or protect authentication credentials, allowing attackers to bypass authentication controls, compromise user accounts, and gain unauthorized access to sensitive data and functionality.

\textbf{Severity}: Critical



\section{CWE-384: Session Fixation}

\textbf{Description}: The BookShop application is vulnerable to session fixation attacks due to the absence of session regeneration after successful authentication. The application uses the same session ID before and after login, allowing attackers to obtain a valid session ID and trick victims into using it, enabling complete session hijacking and unauthorized access to user accounts.

\textbf{CWE Explanation}: CWE-384 occurs when the application fails to regenerate session identifiers after successful authentication, allowing attackers to predict, obtain, or manipulate session IDs to hijack user sessions and gain unauthorized access to sensitive data and functionality.

\textbf{Severity}: High



\end{document}
